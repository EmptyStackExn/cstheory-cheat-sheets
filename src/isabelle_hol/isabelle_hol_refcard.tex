\documentclass[10pt,landscape,a4paper]{article}
\usepackage{multicol}
\usepackage{calc}
\usepackage{ifthen}
\usepackage[landscape]{geometry}
\usepackage{amssymb}
\usepackage{amsmath}
\usepackage{xfrac}
\usepackage{galois}
\usepackage{graphicx}
\usepackage{tikz}
\usetikzlibrary{matrix}

\usepackage{alltt}
\usepackage{stmaryrd}

\renewcommand*{\familydefault}{\sfdefault}


% To make this come out properly in landscape mode, do one of the following
% 1.
%  pdflatex latexsheet.tex
%
% 2.
%  latex latexsheet.tex
%  dvips -P pdf  -t landscape latexsheet.dvi
%  ps2pdf latexsheet.ps


% If you're reading this, be prepared for confusion.  Making this was
% a learning experience for me, and it shows.  Much of the placement
% was hacked in; if you make it better, let me know...


% 2008-04
% Changed page margin code to use the geometry package. Also added code for
% conditional page margins, depending on paper size. Thanks to Uwe Ziegenhagen
% for the suggestions.

% 2006-08
% Made changes based on suggestions from Gene Cooperman. <gene at ccs.neu.edu>


% To Do:
% \listoffigures \listoftables
% \setcounter{secnumdepth}{0}


% This sets page margins to .5 inch if using letter paper, and to 1cm
% if using A4 paper. (This probably isn't strictly necessary.)
% If using another size paper, use default 1cm margins.
\ifthenelse{\lengthtest { \paperwidth = 11in}}
	{ \geometry{top=1cm,left=1cm,right=1cm,bottom=1cm} }
	{\ifthenelse{ \lengthtest{ \paperwidth = 297mm}}
		{\geometry{top=1cm,left=1cm,right=1cm,bottom=1cm} }
		{\geometry{top=1cm,left=1cm,right=1cm,bottom=1cm} }
	}

% Turn off header and footer
\pagestyle{empty}
 

% Redefine section commands to use less space
\makeatletter
\renewcommand{\section}{\@startsection{section}{1}{0mm}%
                                {-1ex plus -.5ex minus -.2ex}%
                                {0.5ex plus .2ex}%x
                                {\normalfont\large\bfseries}}
\renewcommand{\subsection}{\@startsection{subsection}{2}{0mm}%
                                {-1explus -.5ex minus -.2ex}%
                                {0.5ex plus .2ex}%
                                {\normalfont\normalsize\bfseries}}
\renewcommand{\subsubsection}{\@startsection{subsubsection}{3}{0mm}%
                                {-1ex plus -.5ex minus -.2ex}%
                                {1ex plus .2ex}%
                                {\normalfont\small\bfseries}}
\makeatother

% Define BibTeX command
\def\BibTeX{{\rm B\kern-.05em{\sc i\kern-.025em b}\kern-.08em
    T\kern-.1667em\lower.7ex\hbox{E}\kern-.125emX}}

% Don't print section numbers
\setcounter{secnumdepth}{0}


\setlength{\parindent}{0pt}
\setlength{\parskip}{0pt plus 0.5ex}


% -----------------------------------------------------------------------

\begin{document}

\raggedright
\footnotesize
\begin{multicols}{3}


% multicol parameters
% These lengths are set only within the two main columns
%\setlength{\columnseprule}{0.25pt}
\setlength{\premulticols}{1pt}
\setlength{\postmulticols}{1pt}
\setlength{\multicolsep}{1pt}
\setlength{\columnsep}{2pt}

\begin{center}
  \Large{Isabelle/HOL Proof Assistant} %\\
  % \normalsize{Basic Category Theory for Computer Scientists}\\
  \vspace{0.3cm}
  %  \tiny {Inspir\'{e} des notes de cours du MPRI de Paul-Andr\'{e} Melli\`{e}s, CNRS/ENS Ulm}\\
\end{center}

\section{Isar engine}
Use of Natural Deduction (ND) instead of Gentzen sequent calculus (LJ/LK). Thus, $\Gamma$ is a \textbf{metacontext} and its objects can be used with following commands : \verb!assume A!, \verb!show B!\dots

\section{Syntax of Isabelle commands}

\subsubsection{Inspection Commands}
\begin{itemize}
\item type-checking terms : \verb!term "!$\langle\textsf{hol-term}\rangle$\verb!"! 
\item evaluating terms : \verb!value "!$\langle\textsf{hol-term}\rangle$\verb!"! 
\item explore libraries : \verb!find_theorems <theorem content>* [name: <theorem name>]*! 
\end{itemize}

\subsubsection{Specification Commands}
\begin{itemize}
\item non-recursive definitions : \verb!definition f :: "!$\langle\tau\rangle$\verb!"!
  \verb!where !\textsf{name}\verb! : "f x1 ... xn = !$\langle\tau\rangle$\verb!"!
\item type definitions : \verb!typedef ('a1, ..., 'an) !$\kappa$\verb! = "!$\langle\textsf{set-expr}\rangle$\verb!" !$\langle\textsf{proof}\rangle$

  \textbf{Example.} \verb!typedef even  = "{ x :: int. x mod 2 = 0 }"!
\end{itemize}

\subsubsection{Specification Mechanism Commands}
\begin{itemize}
\item datatype definitions:
\begin{eqnarray*}
  \verb!datatype ('a1, ..., 'an) = ! \langle c \rangle\verb! :: "! \langle \tau \rangle \verb!"! | \dots | \langle c \rangle\verb! :: "! \langle \tau \rangle \verb!"! 
\end{eqnarray*}

\item recursive function definitions : \verb!typedef ('a1, ..., 'an) !$\kappa$\verb! = "!$\langle\textsf{set-expr}\rangle$\verb!" !$\langle\textsf{proof}\rangle$

  \textbf{Example.} \verb!typedef even  = "{ x :: int. x mod 2 = 0 }"!

\item inductively defined sets:
\begin{alltt}
inductive <c> [ for <v> :: "<\(\tau\)>" ]
  where <thmname> : "<\(\varphi\)>"
      | ...
      | <thmname> = "<\(\varphi\)>"
\end{alltt}

\textbf{Example.}
\begin{alltt}
inductive path for rel :: "'a \(\Rightarrow\) 'a \(\Rightarrow\) bool" 
where base : "path rel x x"
    | step : "rel x y \(\Rightarrow\) path rel y z \(\Rightarrow\) path rel x z"
\end{alltt}

\item extended notation for cartesian products (records):
\begin{alltt}
record <c> = [<record> +]
  tag\(\sb{1}\) :: "<\(\tau\sb{1}\)>"
  ...
  tag\(\sb{n}\) :: "<\(\tau\sb{n}\)>"
\end{alltt}

\end{itemize}

\section{Apply rules and theorems}

\begin{itemize}
\item $\verb!apply assumption!$ proves $\llbracket B_1, \dots, B_m \rrbracket \Rightarrow C$ by unifying $C$ with one of the $B_i$

\item $\verb!apply (rule !\langle\textsf{intro-rule}\rangle\verb!)!$ :
  \begin{enumerate}
  \item decompose formulae to the right of $\Rightarrow$
  \item applying $\llbracket A_1, \dots, A_n \rrbracket \Rightarrow A$ to subgoal $C$,
    \begin{enumerate}
    \item unify $A$ and $C$
    \item replace $C$ with $n$ new subgoals $A_1$, \dots, $A_n$
    \end{enumerate}
  \end{enumerate}

\item $\verb!apply (erule !\langle\textsf{elim-rule}\rangle\verb!)!$ :
  \begin{enumerate}
  \item decompose formulae to the left of $\Rightarrow$
  \item applying $\llbracket A_1, \dots, A_n \rrbracket \Rightarrow A$ to subgoal $C$, like \textsf{rule} but also
    \begin{enumerate}
    \item unifies first premise of rule with an assumption
    \item eliminates that assumption
    \end{enumerate}
  \end{enumerate}

\item $\verb!apply (case_tac !\langle\textsf{term}\rangle\verb!)!$ : case distinctions on arbitrary terms (e.g. \verb!excluded_middle! on type \textsf{bool})

\item $\verb!apply (rule_tac x = "! \langle \mathsf{term} \rangle \verb!" in ! \langle \mathsf{rule} \rangle \verb!)!$
  \begin{enumerate}
  \item like \textsf{rule} but $? x$ in $\langle \mathsf{rule} \rangle$ is instanciated by $\langle \mathsf{term} \rangle$ before application
  \item applying $\llbracket A_1, \dots, A_n \rrbracket \Rightarrow A$ to subgoal $C$,
    \begin{enumerate}
    \item $x$ is in $\langle \mathsf{rule} \rangle$, not in goal
    \item $\langle \mathsf{term} \rangle$ may contain parameters from the goal and those introduced in Isar texts
    \end{enumerate}
  \end{enumerate}

  \item $\verb!apply (erule_tac x = "! \langle \mathsf{term} \rangle \verb!" in ! \langle \mathsf{rule} \rangle \verb!)!$ : similar

  \item $\verb!apply (rename_tac !x_1 \dots x_n \verb!)!$ renames the rightmost (inner) $n$ parameters to $x_1$, \dots, $x_n$

  \item $\verb!apply (frule !\langle\textsf{rule}\rangle\verb!)!$ : ?
  \item $\verb!apply clarify!$ : applies all safe rules that do not split the goal
  \item $\verb!apply safe!$ : applies all safe rules
  \item $\verb!apply fast!$ : sequent based automatic
  \item $\verb!apply best!$ : search tactics
  \item $\verb!apply blast!$ : automatic tableaux prover (works well on predicate logic)
  \item $\verb!apply metis!$ : resolution prover for FO with equality
  \item $\verb!insert!$ ?
\end{itemize}

\section{Logical rules}

\subsubsection{Rules Safety}
\textbf{Definition.} \textbf{Safe rules} preserve provability.
\begin{itemize}
\item safe: \verb!conjI!, \verb!impI!, \verb!notI!, \verb!iffI!, \verb!refl!, \verb!ccontr!, \verb!classical!, \verb!conjE!, \verb!disjE!, \verb!allI!, \verb!exE!
\item unsafe: \verb!disjl1!, \verb!disjl2!, \verb!impE!, \verb!iffD1!, \verb!iffD2!, \verb!notE!, \verb!allE!, \verb!exI!
\end{itemize}
\begin{enumerate}
\item Apply safe rules before unsafe ones
\item Create parameters first, unknowns later
\end{enumerate}

\subsubsection{Description operator}
\begin{enumerate}
\item $\varepsilon$-operator : $\varepsilon x ~ P(x)$ is a value that satisfies $P$ (if exists) ; written in Isabelle as $\verb!SOME! ~ x. P x$ s.t.
  \begin{eqnarray*}
    \frac{P (? x)}{P (\texttt{SOME} ~ x. P(x))} ~ \texttt{someI}
  \end{eqnarray*}
\item $\iota$-operator : $\varepsilon$ implies Axiom of Choice\footnote{Axiom of Choice : $\forall x ~ \exists y ~ P x y \Rightarrow \exists f ~ \forall x ~ P ~ x ~ (f x)$} :
  
\end{enumerate}

\subsubsection{Intuitionistic logic (LJ)}
\begin{tabular}{c|c|c}
  \textbf{Name}              & \textbf{Rule in Gentzen style} & \textbf{Desc.} \\ \hline \hline
  $\verb!TrueI!$             & $\displaystyle\frac{}{\Gamma \vdash \top}$ &  \\ [2ex]
  $\verb!FalseE!$             & $\displaystyle\frac{\Gamma \vdash \bot}{\Gamma \vdash P}$ &  \\ [2ex]
  $\verb!notI!$              & $\displaystyle\frac{\Gamma, A \vdash \bot}{\Gamma \vdash \neg A}$ &  \\ [2ex]
  $\verb!notE!$              & $\displaystyle\frac{\Gamma \vdash A \quad \Gamma \vdash \neg A}{\Gamma \vdash P}$ &  \\ [2ex]
  $\verb!conjI!$             & $\displaystyle\frac{\Gamma \vdash A \quad \Gamma \vdash B}{\Gamma \vdash A \wedge B}$ &  \\ [2ex]
  $\verb!conjE!$             & $\displaystyle\frac{\Gamma \vdash A \wedge B \quad \Gamma, A, B \vdash C}{\Gamma \vdash C}$ &  \\ [2ex]
  $\verb!conjunct1!$         & $\displaystyle\frac{\Gamma \vdash A \wedge B}{\Gamma \vdash A}$ &  \\ [2ex]
  $\verb!conjunct2!$         & $\displaystyle\frac{\Gamma \vdash A \wedge B}{\Gamma \vdash B}$ &  \\ [2ex]
  $\verb!context_conjI!$     & $\displaystyle\frac{\Gamma \vdash A \quad \Gamma, A \vdash B}{\Gamma \vdash A \wedge B}$ &  \\ [2ex]
  $\verb!disjI1!$            & $\displaystyle\frac{\Gamma \vdash A}{\Gamma \vdash A \vee B}$ &  \\ [2ex]
  $\verb!disjI2!$            & $\displaystyle\frac{\Gamma \vdash A}{\Gamma \vdash B \vee A}$ &  \\ [2ex]
  $\verb!disjCI!$            & $\displaystyle\frac{\Gamma, \neg B \vdash A}{\Gamma \vdash A \vee B}$ &  \\ [2ex]
  $\verb!disjE!$             & $\frac{\Gamma \vdash A \vee B \quad \Gamma, A \vdash C \quad \Gamma, B \vdash C}{\Gamma \vdash C}$ &  \\ [2ex]
  $\verb!impI!$              & $\displaystyle\frac{\Gamma, A \vdash B}{\Gamma \vdash A \Rightarrow B}$ &  \\ [2ex]
  $\verb!impE!$              & $\frac{\Gamma \vdash A \Rightarrow B \quad \Gamma \vdash A \quad \Gamma, B \vdash C}{\Gamma \vdash C}$ &  \\ [2ex]
  $\verb!impCE!$             & $\displaystyle\frac{\Gamma, A \vdash B \quad \Gamma, \neg A \vdash B}{\Gamma \vdash B}$ &  \\ [2ex]
  $\verb!mp!$                & $\displaystyle\frac{\Gamma \vdash A \Rightarrow B \quad \Gamma \vdash A}{\Gamma \vdash B}$ & $\Rightarrow$-elim \\ [2ex]
  $\verb!iffI!$              & $\frac{\Gamma \vdash A \Rightarrow B \quad \Gamma \vdash B \Rightarrow A}{\Gamma \vdash A \Leftrightarrow B}$ &  \\ [2ex]
  $\verb!iffE!$              & $\frac{\Gamma \vdash A \Leftrightarrow B \quad \Gamma, A \Rightarrow B, B \Rightarrow A \vdash C}{\Gamma \vdash C}$ &  \\ [2ex]
  $\verb!iffD1!$              & $\displaystyle \frac{\Gamma \vdash A \Leftrightarrow B}{\Gamma \vdash A \Rightarrow B}$ &  \\ [2ex]
  $\verb!iffD2!$              & $\displaystyle \frac{\Gamma \vdash A \Leftrightarrow B}{\Gamma \vdash B \Rightarrow A}$ & 
\end{tabular}

\subsubsection{De Morgan Extensions}
\begin{tabular}{c|c|c}
  $\verb!notnotD!$           & $\displaystyle\frac{\Gamma, \neg \neg P}{\Gamma \vdash P}$ &  \\ [2ex]
  $\verb!de_Morgan_disj!$     & $\neg (A \vee B) \Leftrightarrow \neg A \wedge \neg B$ &  \\ [2ex]
  $\verb!de_Morgan_conj!$     & $\neg (A \wedge B) \Leftrightarrow \neg A \vee \neg B$ &  \\ [2ex]
  $\verb!disj_not1!$          & $\neg P \vee Q \Leftrightarrow P \Rightarrow Q$ &  \\ [2ex]
  $\verb!disj_not2!$          & $P \vee \neg Q \Leftrightarrow Q \Rightarrow P$ &  
\end{tabular}

\subsubsection{Non-Constructive Classical Logic (LK)}
\begin{tabular}{c|c|c}
  \textbf{Name}   & \textbf{Rule in sequent style} & \textbf{Desc.} \\ \hline \hline
  $\verb!True_or_false!$     & $\displaystyle\frac{}{\Gamma \vdash A \Leftrightarrow \top \vee A \Leftrightarrow \bot}$ &  \\ [2ex]
  $\verb!classical!$         & $\displaystyle\frac{\Gamma, \neg A \vdash A}{\Gamma \vdash A}$ & absurd ? \\ [2ex]
  $\verb!excluded_middle!$   & $\displaystyle\frac{}{\Gamma \vdash A \vee \neg A}$ &  \\ [2ex]
  $\verb!ccontr!$            & $\displaystyle\frac{\Gamma, \neg A \vdash \bot}{\Gamma \vdash \bot}$ &  \\ [2ex]
\end{tabular}


\subsubsection{Non-Constructive First Order Logic (FO)}
\begin{itemize}
\item $\bigwedge x$ : new parameters introduced
\item $? x$ : new unknowns introduced
\end{itemize}

\begin{tabular}{c|c|c}
  \textbf{Name}   & \textbf{Rule in sequent style} & \textbf{Desc.} \\ \hline \hline
  $\verb!allI!$   & $\displaystyle\frac{\Gamma \vdash \bigwedge x ~ P (x)}{\Gamma \vdash \forall x~ P(xà}$ & $\forall$-intro \\ [2ex]
  $\verb!allE!$   & $\displaystyle\frac{\Gamma \vdash \forall x~ P \quad \Gamma, P(? x) \vdash Q}{\Gamma \vdash Q}$ &  \\ [2ex]
  $\verb!exI!$    & $\displaystyle\frac{\Gamma \vdash P (? x)}{\Gamma \vdash \exists x~ P (x)}$ & $\exists$-intro\\
  $\verb!exE!$    & $\displaystyle\frac{\Gamma \vdash \exists x ~ P \quad \Gamma, \bigwedge x ~ P(x) \vdash Q}{\Gamma \vdash Q}$ & $\exists$-elim \\ [2ex]

  $\verb!spec!$   & $\displaystyle\frac{\Gamma \vdash \forall x~ P(x)}{\Gamma \vdash P (? x)}$ &  \\ [2ex]

\end{tabular}

\subsubsection{Equational Logic}
\begin{tabular}{c|c|c}
  \textbf{Name}   & \textbf{Rule in sequent style} & \textbf{Desc.} \\ \hline \hline
  $\verb!refl!$    & $\displaystyle\frac{}{x = x}$ & \\ [2ex]
  $\verb!sym!$    & $\displaystyle\frac{y = x}{x = y}$ & \\ [2ex]
  $\verb!trans!$  & $\displaystyle\frac{x = y \quad y = z}{x = z}$ & \\ [2ex]
  $\verb!subst!$  & $\displaystyle\frac{x = y \quad P (x)}{P (y)}$ & \\ [2ex]
  $\verb!ext!$  & $\displaystyle\frac{\wedge t~ P (t) = Q (t)}{P = Q}$ & \\ [2ex]
\end{tabular}

%%%%%%%%%%%%%%%%%%%%%%%%%%%%%%%%%%%%%%%%%%%%%%%%%%%%%%%%%%%%%%%%%%%%%%%%%%%%%%%%%%%%


\rule{0.3\linewidth}{0.25pt}
\scriptsize

\LaTeX \ cheat sheet template made by Winston Chang

http://www.stdout.org/$\sim$winston/latex/


\end{multicols}
\end{document}
